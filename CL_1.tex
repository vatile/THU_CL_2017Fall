%!TEX program = xelatex

%%%%%%%%%%%%%%%%%%%%%%%%%%%%%%%%%%%%%%%%%
% Beamer Presentation
% LaTeX Template
% Version 1.0 (10/11/12)
%
% This template has been downloaded from:
% http://www.LaTeXTemplates.com
%
% License:
% CC BY-NC-SA 3.0 (http://creativecommons.org/licenses/by-nc-sa/3.0/)
%
%%%%%%%%%%%%%%%%%%%%%%%%%%%%%%%%%%%%%%%%%

%----------------------------------------------------------------------------------------
%	PACKAGES AND THEMES
%----------------------------------------------------------------------------------------

\documentclass[xcolor={usenames, dvipsnames}, hyperref={colorlinks,linkcolor=black, urlcolor=blue}]{beamer}

\mode<presentation> {

% The Beamer class comes with a number of default slide themes
% which change the colors and layouts of slides. Below this is a list
% of all the themes, uncomment each in turn to see what they look like.

%\usetheme{default}
%\usetheme{AnnArbor}
%\usetheme{Antibes}
%\usetheme{Bergen}
%\usetheme{Berkeley}
% \usetheme{Berlin}
%\usetheme{Boadilla}
%\usetheme{CambridgeUS}
%\usetheme{Copenhagen}
%\usetheme{Darmstadt}
%\usetheme{Dresden}
%\usetheme{Frankfurt}
%\usetheme{Goettingen}
%\usetheme{Hannover}
%\usetheme{Ilmenau}
%\usetheme{JuanLesPins}
%\usetheme{Luebeck}
\usetheme{Madrid}
%\usetheme{Malmoe}
%\usetheme{Marburg}
%\usetheme{Montpellier}
%\usetheme{PaloAlto}
%\usetheme{Pittsburgh}
%\usetheme{Rochester}
%\usetheme{Singapore}
%\usetheme{Szeged}
%\usetheme{Warsaw}

% As well as themes, the Beamer class has a number of color themes
% for any slide theme. Uncomment each of these in turn to see how it
% changes the colors of your current slide theme.

% \usecolortheme{albatross} % all dark blue
% \usecolortheme{beaver} % white & red
% \usecolortheme{beetle} % all grey
% \usecolortheme{crane} % white & orange
\usecolortheme{dolphin} % white & light purple
% \usecolortheme{dove} % white & black
% \usecolortheme{fly} % all grey
% \usecolortheme{lily} % white & blue
% \usecolortheme{orchid} % white & blue shadow
% \usecolortheme{rose} % white & blue shadow
% \usecolortheme{seagull} % light grey
% \usecolortheme{seahorse} % light purple
% \usecolortheme{whale} % white & blue shadow
% \usecolortheme{wolverine} % bright yellow

%\setbeamertemplate{footline} % To remove the footer line in all slides uncomment this line
%\setbeamertemplate{footline}[page number] % To replace the footer line in all slides with a simple slide count uncomment this line

%\setbeamertemplate{navigation symbols}{} % To remove the navigation symbols from the bottom of all slides uncomment this line
}

\usepackage{graphicx} % Allows including images
\usepackage{booktabs} % Allows the use of \toprule, \midrule and \bottomrule in tables


\usepackage{pythonhighlight}
\usepackage{xeCJK}
%\setCJKsansfont{黑体}
%----------------------------------------------------------------------------------------
%	TITLE PAGE
%----------------------------------------------------------------------------------------

\title[Computational Linguistics \emph{Practical}]{Basic Data Types and Sequence Operations in Python} % The short title appears at the bottom of every slide, the full title is only on the title page
\author{YE Yuxiao} % Your name
\institute[] % Your institution as it will appear on the bottom of every slide, may be shorthand to save space
{
Tsinghua University \\ % Your institution for the title page
\medskip
\textit{yeyuxiao@outlook.com} % Your email address
}
\date{2017/09/29} % Date, can be changed to a custom date

\begin{document}

\begin{frame}
\titlepage % Print the title page as the first slide
\end{frame}

\begin{frame}
\frametitle{Overview} % Table of contents slide, comment this block out to remove it
\tableofcontents % Throughout your presentation, if you choose to use \section{} and \subsection{} commands, these will automatically be printed on this slide as an overview of your presentation
\end{frame}

%----------------------------------------------------------------------------------------
%	PRESENTATION SLIDES
%----------------------------------------------------------------------------------------

%------------------------------------------------
\section{Python - What and Why} % Sections can be created in order to organize your presentation into discrete blocks, all sections and subsections are automatically printed in the table of contents as an overview of the talk
%------------------------------------------------

\subsection{What is Python} % A subsection can be created just before a set of slides with a common theme to further break down your presentation into chunks
\begin{frame}
\frametitle{What is Python}

\begin{itemize}
\item A high level programming language
\item Similar to natural languages
\item Easy to learn
\end{itemize}
\end{frame}

%------------------------------------------------

\subsection{Why should we learn Python} 
\begin{frame}
\frametitle{Why should we learn Python}

\begin{itemize}
\item Fun. Seriously.
\item Power and control!
\item It can help you get a job.
\end{itemize}

\begin{flushright}
\texttt{--} Korin Richmond
\end{flushright} 
\end{frame}

%----------------------------------------------------------------------------------------

%------------------------------------------------
\section{Basic Data Types in Python}
%------------------------------------------------
\begin{frame}
\frametitle{Basic Data Types in Python}

\begin{itemize}
\item Numbers
\item Strings
\item Lists
\item Tuples
\item Dictionaries
\end{itemize}
\end{frame}

%------------------------------------------------
\subsection{Numbers} 
\begin{frame}
\frametitle{Numbers}

\begin{itemize}
\item Integers (int): a very self-explanatory name
	\begin{itemize}
	\item e.g. $0$, $1$, $-2896$
	\end{itemize}
\end{itemize}

\begin{itemize}
\item Floating Point (float): numbers with decimal points
	\begin{itemize}
	\item e.g. $0.1$, $1.0$, $-3.1415926$
	\end{itemize}
\end{itemize}

\begin{itemize}
\item Mathematical operations
	\begin{itemize}
	\item try \pyth{print(6+4)}
	\end{itemize}
\end{itemize}
\end{frame}

%------------------------------------------------
\subsection{Strings} 
\begin{frame}
\frametitle{Strings}

\begin{itemize}
\item Extremely superficial (only appearance matters)
	\begin{itemize}
	\item e.g. "print", "100", "6+4"
	\item try \pyth{print("6+4")} and \pyth{"print"(6+4)}
	\end{itemize}
\end{itemize}

\begin{itemize}
\item Anything inside a pair of single/double quotes will be treaded as a string.
	\begin{itemize}
	\item '6+4'
	\item "6+4"
	\end{itemize}
\end{itemize}

\end{frame}

%------------------------------------------------


\begin{frame}
\frametitle{Strings}

\begin{itemize}
\item An unpaired quote always tries to pair with its nearest succeeding unpaired quote.
	\begin{itemize}
	\item '"6+4", she said'
	\item "'6+4', she said"
	\end{itemize}
\end{itemize}

\begin{itemize}
\item The backslash "\textbackslash": a quote right behind a "\textbackslash" will lose its syntactic function; "{\textbackslash}n" is a newline symbol.
	\begin{itemize}
	\item "{\color{red}\textbackslash"}6+4{\color{red}\textbackslash"}, she said"
	\item \pyth {print('Hello \\nworld')}
	\end{itemize}
\end{itemize}

\end{frame}

%------------------------------------------------

\subsection{Lists}
\begin{frame}
\frametitle{Lists}

\begin{itemize}
\item A sequence (\textbf{ordered}) of items included in a pair of brackets: [ ].
	\begin{itemize}
	\item $[$ $]$
	\item $[$1, 2, 3, 4$]$
	\item $[$4, 3, 2, 1$]$
	\item $[$'a', 'b', 'c'$]$
	\end{itemize}
\end{itemize}

\begin{itemize}
\item A left bracket always tries to pair with its furtherest succeeding right bracket.
	\begin{itemize}
	\item $[$$[$1, 2$]$, 3, "4"$]$
	\end{itemize}
\end{itemize}

\end{frame}

%------------------------------------------------

\subsection{Tuples}

\begin{frame}
\frametitle{Tuples}

\begin{itemize}
\item A sequence of items included in a pair of parentheses: ( ).
	\begin{itemize}
	\item $($$($1, 2$)$, $[$3, "4"$]$$)$
	\end{itemize}
\end{itemize}

\begin{itemize}
\item Tuples are very similar to lists, except they are \textbf{immutable}. Strings are also \textbf{immutable}.
	\begin{itemize}
	\item \pyth {a = [1, 2, 3, 4]}
	\item \pyth {b = (1, 2, 3, 4)}
	\item \pyth {c = "1, 2, 3, 4"}
	\item \pyth {a.append(5)}
	\item \pyth {print(a)}
	\item see what happens when \pyth {b.append(5)} and \pyth {c.append('5')}
	\end{itemize}
\end{itemize}

\end{frame}

%------------------------------------------------

\subsection{Dictionaries} 
\begin{frame}
\frametitle{Dictionaries}

\begin{itemize}
\item Items stored in lists and tuples are \textbf{ordered}. You can always find an item in a list/tuple with their \textbf{position}.
	\begin{itemize}
	\item e.g. the first item in $[$1, 2, 3$]$ is 1; the first item in $[$3, 2, 1$]$ is 3
	\end{itemize}
\end{itemize}

\begin{itemize}
\item In dictionaries, items are stored (not ordered) with a name, in the form $\rm \left\{{\color{red}key_1:\ value_1}, \ {\color{CornflowerBlue}key_2:\ value_2},\ ...\right\}$. 
\end{itemize}

\end{frame}

%------------------------------------------------


\begin{frame}
\frametitle{Dictionaries}

\begin{itemize}
\item Keys can be any immutable objects (e.g., numbers, strings, tuples, but not lists or dictionaries). Values can be any objects.
	\begin{itemize}
	\item $\{$ $\}$
	\item $\{${\color{red}'a': 1}, {\color{Dandelion}(1, 2): '2'}, {\color{CornflowerBlue}$\{$1: 'a'$\}$: $[$1, 2$]$}$\}$
	\end{itemize}
\end{itemize}

\begin{itemize}
\item Get/change a value by querying the dictionary with its key.
	\begin{itemize}
	\item \pyth {d = \{'a': 1, (1, 2): '2', \{1: 'a'\}: [1, 2]\}}
	\item \pyth {print(d[(1, 2)])}
	\item \pyth {d[(1, 2)] = 3}
	\item \pyth {print(d)}
	\end{itemize}
\end{itemize}

\end{frame}

%----------------------------------------------------------------------------------------

%------------------------------------------------
\section{Sequence Operations}
%------------------------------------------------
\begin{frame}
\frametitle{Sequence Operations}

Strings, lists, and tuples are all sequence objects (sets of ordered items) in Python. 

\begin{itemize}
\item Indexing
\item Slicing
\end{itemize}
\end{frame}

%------------------------------------------------

\subsection{Indexing} 
\begin{frame}
\frametitle{Indexing}

\begin{itemize}
\item Each Item is indexed by their position (represented as integers).
\end{itemize}

\begin{itemize}
\item \textbf {ATTENTION}: in Python we always count from 0! So the first item in a sequence has an index of 0.
	\begin{itemize}
	\item \pyth {l = ["a", "b", "c", "d"]}
	\item \pyth {s = "Hello world"}
	\item \pyth {print(l[0])}
	\item \pyth {print(s[-1])}
	\item how many characters in s? what is s$[5]$?
	\end{itemize}
\end{itemize}
\end{frame}

%------------------------------------------------

\subsection{Slicing} 
\begin{frame}
\frametitle{Slicing}

\begin{itemize}
\item A slice is a sub-sequence of a sequence object.
\end{itemize}

\begin{itemize}
\item Syntax for slicing: sequence[start:end:step]. The start index is inclusive; the end index is exclusive.
	\begin{itemize}
	\item \pyth {s = "Hello world"}
	\item \pyth {print(s[1:6:2])}
	\item \pyth {print(s[6:1:-1])}
	\end{itemize}
\end{itemize}

\begin{itemize}
\item If the start/end index is not specified, its default value 0/-1 will be used. If the step index is not specified, its default value 1 will be used, and the colon before it can be omitted.
	\begin{itemize}
	\item \pyth {s = "Hello world"}
	\item \pyth {print(s[::2])}
	\item \pyth {print(s[1:6])}
	\item try \pyth {print(s[6:1])}
	\end{itemize}
\end{itemize}

\end{frame}

%------------------------------------------------

\begin{frame}
\frametitle{Practice}

Try to solve these problems using Python. You are \textbf{soooo} allowed to search the internet. Getting help from online sources is a crucial skill for programmers. 

\begin{enumerate}
\item Create a list containing three names of you friends, and assign it to a variable called MyFriend.
\item Make items in MyFriend ordered alphabetically.
\item Reverse MyFriend.
\end{enumerate}
\end{frame}

%------------------------------------------------

\begin{frame}
\frametitle{Q \& A}

If you have any questions, you can either post on Piazza, or:

\begin{itemize}
\item Check Python Documentation: \url{https://docs.python.org/3.5/tutorial/index.html}
\item Search Stack Overflow: \url{https://stackoverflow.com}
\item Ask Google/Baidu/Bing...
\item ...
\end{itemize}
\end{frame}

%----------------------------------------------------------------------------------------

\end{document} 